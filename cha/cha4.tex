\chapter{激光的基本技术}
\section{激光器输出的选模}
\section{激光器的稳频}
\section{激光束的变换}
\section{激光调制技术}
\section{激光偏转技术}
\section{激光调Q技术}
\section{激光锁模技术}
\newpage
\section*{Problems}
\begin{problem}
	Gaussian光束束腰半径$\omega_0=\SI{0.2}{\mm},\ \lambda=\SI{0.6328}{\um}$. 今用一焦距$f=\SI{3}{\cm}$的短焦距透镜聚焦,已知腰粗$\omega_0$离透镜的距离$s=\SI{60}{\cm}$,在几何光学近似下求聚焦后光束腰粗.
\end{problem}
\begin{solution}
由于短焦距条件下聚焦后光束腰粗(4.31)得
\[\omega_0'=\frac{\lambda f}{\pi\omega_0\sqrt{1+\ab(\dfrac{\lambda s}{\pi\omega_0^2})^2}}=\SI{9.49}{\um}\]
考虑到$\ab(\frac{\lambda s}{\pi\omega_0^2})^2=9.13\gg 1$,所以亦可使用(4.31)的简化(4.33)
\[\omega_0'\approx\frac{f\omega_0}{s}=\SI{0.01}{\mm}\]
此近似下相对误差$\eta=5.33\%$.
\end{solution}

\begin{paracol}{2}
\begin{problem}
    用如图所示的倒置望远镜系统改善由对称共焦腔输出的光束方向性. 已知两个透镜的焦距分别为$f_1=\SI{2.5}{\cm},\ f_2=\SI{20}{\cm},\ \lambda=\SI{0.6328}{\um},\ \omega_0=\SI{0.28}{\mm},\ l_1\gg f_1$($L_1$紧靠腔的输出镜面). 求该望远镜系统光束发散角的压缩比.
\end{problem}
\switchcolumn\centering
\vfill
\begin{tikzpicture}

\end{tikzpicture}
\vfill
\end{paracol}
\begin{solution}
    
\end{solution}